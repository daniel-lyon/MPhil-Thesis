\chapter{Decomposed Infrared Luminosity Functions of SFG and AGN}

\begin{adjustwidth}{2cm}{0cm}
    This chapter is based on the work published in \cite{lyon_decomposing_2024}. The co-authors are M. Cowley, O. Pye, and A. Hopkins. M. Cowley supervised this work and contributed to sections \textcolor{red}{2 and 3}. O. Pye performed the decomposition and statistical analysis. A. Hopkins provided support and useful discussions on luminosity functions. The candidate's contribution to the work presented is 95\%.
\end{adjustwidth}

\section{Abstract}
This study presents a comprehensive analysis of the infrared (IR) luminosity functions (LF) of star-forming (SF) galaxies and active galactic nuclei (AGN) using data from the ZFOURGE survey. We employ CIGALE to decompose the spectral energy distribution (SED) of galaxies into SF and AGN components to investigate the co-evolution of these processes at higher redshifts and fainter luminosities. Our CIGALE-derived SF and AGN LFs are generally consistent with previous studies, with an enhancement at the faint end of the AGN LFs. We attribute this to CIGALE's capability to recover low-luminosity AGN more accurately, which may be underrepresented in other works. As anticipated, the CIGALE SF LFs are best fit with a Schechter function, whereas the AGN LFs align more closely with a Saunders function. We find evidence for a significant evolutionary epoch for AGN activity at $z \approx 1$, comparable to the peak of cosmic star formation at $z \approx 2$, which we also recover well. Based on our results, the gas supply in the early universe favoured the formation of brighter star-forming galaxies until $z=2$, below which the gas for SF becomes increasingly exhausted. Conversely, AGN activity peaked earlier and declined more slowly until $z \approx 1$, suggesting a possible feedback scenario in which $2.5-3$ Gyrs offset the evolution of SF and AGN activity. 