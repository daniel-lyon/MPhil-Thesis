\section{Introduction}
The distribution of galaxies and their luminosities have previously derived powerful constraints on galaxy evolution \citep{binggeli_luminosity_1988, benson_what_2003, rodighiero_mid-_2010, gruppioni_herschel_2013}. One of the direct ways of measuring the distribution of galaxies is with the luminosity function \citep{schechter_analytic_1976, saunders_60-mum_1990}. Luminosity functions (LFs) are statistical distributions that describe the spatial density of astronomical objects and are a fundamental tool for quantifying their evolution across cosmic time scales \citep{dai_mid-infrared_2009, han_evolution_2012, wylezalek_galaxy_2014}. The use of LFs in galaxy evolution studies has uncovered a wealth of information revealing the intricate processes governing star formation (SF), galaxy mergers, and the growth of supermassive black holes (SMBH, $M_{BH} > 10^{6}\ M_{\odot}$) across cosmic time \citep{caputi_infrared_2007, hopkins_observational_2007, magnelli_deepest_2013, delvecchio_tracing_2014, hernan-caballero_resolving_2015}. Specifically, many such studies find a strong correlation between the activity of the central SMBH and the star formation rate (SFR) \citep{hopkins_cosmological_2008, merloni_synthesis_2008}. 

Active galactic nuclei (AGN) are actively accreting SMBHs, whereby massive quantities of gas and dust power their growth \citep{hopkins_cosmological_2008, han_evolution_2012, toba_9_2013, brown_infrared_2019}. It is widely accepted that most galaxies, particularly those with significant bulges, host a SMBH at their centre \citep{gruppioni_modelling_2011, han_evolution_2012, brown_infrared_2019}. While not all central SMBHs are currently active, such as Sagittarius A$^{*}$ at the centre of our own Milky Way \citep{event_horizon_telescope_collaboration_first_2022}, most galaxies have likely experienced the influence of an AGN at some point in their history \citep{gruppioni_modelling_2011}. SF in galaxies similarly requires an extensive reservoir of cool gas to operate \citep{schawinski_observational_2007, cicone_massive_2014}. The same interstellar material that powers SF can also fuel AGN growth, leading to an inherent link between these two processes \citep{hopkins_cosmological_2008, brown_infrared_2019}. This connection has fuelled an ongoing debate over whether AGN activity enhances SF by triggering gas inflows or diminishes it through feedback mechanisms that deplete gas reservoirs \citep{grazian_galaxy_2015, fiore_agn_2017}.

Understanding the role of AGN in galaxy evolution is essential, as these processes regulate the growth of the SMBH and the host galaxy's development. Studies have shown that the most luminous AGN are often preceded by periods of intense SF \citep{kauffmann_host_2003, hopkins_cosmological_2008, hopkins_how_2010} suggesting a co-evolutionary relationship. Some of this activity is seen as AGN-driven outflow winds that can expel the interstellar medium (ISM) from the galaxy \citep{schawinski_observational_2007, cicone_massive_2014, fiore_agn_2017}, thereby starving the galaxy of the cold gas needed for both SF and AGN fueling, ultimately shutting down both processes \citep{hopkins_how_2010}. However, \cite{silk_unleashing_2013} suggested that AGN-driven winds may, under certain conditions, compress gas and dust, thereby enhancing SF. This scenario aligns with findings from \citet{cowley_zfourge_2016}, which indicate that AGN-dominated systems tend to have higher specific star formation rates, suggesting that SF and AGN activity can co-exist in certain environments.

Both SF and AGN activity releases an enormous amount of energy across the entire electromagnetic spectrum, from radio waves to gamma rays \citep{ho_spectral_1999, huang_local_2007, silva_modelling_2011, gruppioni_modelling_2011}. LFs at various wavelengths have been used to place powerful constraints on evolutionary models, as seen in \cite{aird_evolution_2015, alqasim_new_2023} (X-ray), \cite{yuan_determining_2018} (Radio), \cite{page_ultraviolet_2021} (UV), and \cite{cool_galaxy_2012} (optical). However, the properties of Infrared (IR) light make it the ideal regime for studying SF and AGN LFs as both processes are often dusty and obscured (optically thick) \citep{wu_mid-infrared_2011, han_evolution_2012}. Dust extinction absorbs the outgoing X-ray, optical, and UV wavelengths and re-emits the radiation in the IR domain \citep{fu_decomposing_2010, toba_9_2013, oconnor_luminosity_2016, symeonidis_agn_2021}. AGN activity is closely correlated with IR luminosity because the IR emission often traces dust heated by AGN \citep{kauffmann_host_2003, wu_mid-infrared_2011, symeonidis_what_2019, symeonidis_agn_2021}. However, this can create a bias against detecting faint AGN. This bias may also explain why AGN feedback appears to be more prevalent at lower redshifts \citep{katsianis_evolution_2017, pouliasis_obscured_2020}. The relationship between IR luminosity and SFR is complex \citep{symeonidis_agn_2021} and becomes weaker at brighter luminosities and higher redshifts, where AGN contribution to the IR emission increases \citep{wu_mid-infrared_2011}. 

Most studies focus on galaxy or SF LFs, which trace the evolution of galaxies \citep{tempel_tracing_2011, cool_galaxy_2012}, but often neglect the co-evolution with AGN \citep{fotopoulou_5-10_2016, symeonidis_agn_2021, finkelstein_coevolution_2022} even though galaxies are known to be significantly influenced by this relationship \citep{hopkins_cosmological_2008, fiore_agn_2017}. The population of obscured, dusty IR AGN and SFGs play a crucial role in constraining galaxy evolution models \citep{gruppioni_modelling_2011}. Analysing the SF and AGN LFs is paramount to understanding the complex processes driving galaxy evolution. SF LFs allow us to quantify the cosmic star formation rate history and the star-formation rate density. At the same time, the AGN LF provides insight into gas reservoirs not utilised by SF, particularly in the context of feedback mechanisms.

Since SF and AGN activity are so tightly coupled, distinguishing between the two components can be challenging. Traditional methods of selecting AGN, such as colour-colour diagnostics \citep{lacy_obscured_2004}, X-ray dominated \citep{szokoly_chandra_2004}, and radio-dominated \citep{rees_radio_2016} approaches, often overlook faint AGN due to biases inherent in their selection criteria \citep{thorne_deep_2022}. This limitation highlights the need for more refined methods of identifying AGN to better constrain galaxy evolution. In this paper, we use \texttt{Code Investigating Galaxy Emission} (CIGALE, \citealp{burgarella_star_2005, noll_analysis_2009, boquien_cigale_2019}) to decompose the IR Spectral Energy Distribution (SED) of ZFOURGE galaxies \citep{straatman_fourstar_2016} to generate and analyse both the SF and AGN LFs. Incorporating SED decomposition is crucial for interpreting the co-evolution of galaxies with AGN. SEDs characterise many ongoing processes such as the SF and AGN components, as well as dust attenuation and gas heating \citep{ho_spectral_1999, huang_local_2007, silva_modelling_2011, gruppioni_modelling_2011}. We focus on the SF and AGN components to probe galaxy evolution directly. SED decomposition will allow us to independently quantify SF and AGN evolution while minimising the bias against faint AGN.

To achieve this, we leverage data from the ZFOURGE survey, which probes galaxies at higher redshifts and fainter luminosities, allowing for more precise constraints on the evolution of the SFR and AGN activity across cosmic time. By decomposing the SED of each galaxy, we can disentangle the contributions of SF and AGN to galaxy evolution, providing a clearer picture of how these processes interact. Our approach minimises biases against faint AGN and improves the accuracy of the derived LFs. In section \ref{Sec: The ZFOURGE Survey}, we introduce the ZFOURGE Survey and an overview of our data generation and reduction. In section \ref{Sec: CIGALE}, we overview CIGALE and how we performed the SED decomposition process to disentangle SF and AGN luminosities. In section \ref{Sec: Luminosity Functions}, we provide a brief outline of how we calculate the LFs, the errors, and the functional fitting process. Finally, in section \ref{Sec: Discussion}, we discuss the results of our LFs, the parameter evolution, luminosity density, and luminosity class evolution in the broader context of galaxy co-evolution with AGN and SFR. Throughout this paper we adopt a cosmology of $H_0 = 70$ km $\mathrm{s^{-1}\ Mpc^{-1}}$, $\Omega_m=0.7$, and $\Omega_\Lambda=0.3$.

