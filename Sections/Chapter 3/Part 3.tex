\section{Spectral Luminosity Functions}

In practice, galaxy luminosity is calculated from the \textit{flux} measurement. Flux is measured in $W/m^2$ which is the amount of energy received per second over a given area. However, observing telescopes like the \textit{Spitzer Space Telescope} measure \textit{flux density} through filters that only capture light in specific frequency bands and is measured in $\mu Jy$. $\mu Jy$ can be convered to $W/m^2/Hz$ with. \textit{Spectral luminosity} is the luminosity of a particular filter whereas \textit{bolometric luminosity} refers to total luminosity across all wavelengths. 

\begin{equation}
    F [W/m^2/Hz] = 0.3631 \times F [\mu Jy] \times 10^{-32} 
    \label{EQ: }
\end{equation}
\myequations{Convert $\mu Jy$ to $W/m^2/Hz$}

The \gls{zfourge} measures the spectral flux in many bands mentioned in \cref{Sec: ZFOURGE Overview}. We first test our method of calculating \gls{lf} in the $K_{s}$ band. This requires a different method of calculating the luminosity function as a few extra steps are required to meet the completeness of the survey. Firstly, the magnitude limit in the $K_{s}$ is $K_{s,lim} = 25.9\ AB$. AB magnitudes can be converted to $W/m^2/Hz$ with the following formula:

\begin{equation}
    F = 10^{\frac{M_{AB}+56.1}{-2.5}}
\end{equation}
\myequations{Convert AB magnitude to $W/m^2/Hz$}