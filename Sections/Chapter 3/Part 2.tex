\section{Fitting Functions}
To model LFs, one of the most widely used methods is the Schechter function \citep{schechter_analytic_1976}. This function is beneficial for describing the LF of galaxies because it can represent observed features such as a power-law decline at the faint end and an exponential cutoff at the bright end. We employ the Schechter function to model the CIGALE SF LF as both \cite{fu_decomposing_2010} and \cite{wu_mid-infrared_2011} have shown that pure SF LFs are fit better. The Schechter function is mathematically represented by equation \ref{EQ: Shechter Function}:

\begin{equation} 
    \varphi(L) = \varphi^* \left(\frac{L}{L^*}\right)^{1-\alpha} \exp\left(-\frac{L}{L^*}\right) 
    \label{EQ: Shechter Function}
\end{equation}

Where $\varphi(L)$ is the number of galaxies per unit volume (number density), $\varphi^*$ is the characteristic normalisation factor, $L$ is the bolometric IR (8-1000$\mu$m) luminosity, $L^*$ is the characteristic luminosity, and $\alpha$ is the faint end slope \citep{schechter_analytic_1976}. The Schechter function, however, is not the only commonly used fitting function at mid- and far-IR wavelengths. The bright end slope of the Schechter function cannot be independently varied to better fit a dataset. We make use of a modified Schechter function known as the Saunders function (\citealp{saunders_60-mum_1990}; equation. \ref{EQ: Saunders Function}) to fit our ZFOURGE total and CIGALE AGN LFs:

\begin{equation} 
    \varphi(L) = \varphi^* \left(\frac{L}{L^*}\right)^{1-\alpha} \exp\left[-\frac{1}{2\sigma^2}\log_{10}^2\left(1+\frac{L}{L^*}\right)\right]
    \label{EQ: Saunders Function}
\end{equation}

Where the parameters are the same as the Schechter function (equation \ref{EQ: Shechter Function}), but with the introduction of $\sigma$ to vary the bright end slope. Our deep ZFOURGE data probes to fainter luminosities than often seen in the literature \citep{gruppioni_herschel_2013, rodighiero_mid-_2010}, thus better constraining the faint end of the LF. However, as ZFOURGE is designed to probe deeper into the universe, we lack sources at brighter luminosities. At our lowest redshift bins, we do not have comparatively high luminosity bins to fit the bright end of the LF. We fix $\sigma$ with values matching the bright end of the literature. We opt to fix $\alpha=1.3$ for the ZFOURGE LF and $\alpha=1.2$ our CIGALE LFs as these values better fit our data and are in good agreement with literature. We leave $L^{*}$ and $\phi^{*}$ as free parameters as done in the literature as well. The evolution of $L^{*}$ and $\phi^{*}$ will be discussed in section  where we show the evolution in figure and values in tables and  for ZFOURGE and CIGALE respectively.