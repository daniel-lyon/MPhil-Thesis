\section{Future Directions} \label{Sec: Future Directions}
Future advancements in telescopes, software, and analysis techniques will be critical in addressing the current limitations in IR luminosity function studies, particularly in AGN/SF decomposition and constraints at high redshifts. The advent of next-generation observatories, such as the James Webb Space Telescope (JWST) and the Square Kilometre Array (SKA), will enable deeper and more complete surveys, allowing for better characterisation of the AGN and SF contributions to the infrared output of galaxies. JWST’s superior resolution and sensitivity in the mid-IR (e.g., with MIRI) will enable more precise SED fitting by resolving degenerate AGN and SF components, particularly in highly obscured galaxies. This will refine AGN corrections and improve the accuracy of AGN and SF luminosity functions. Similarly, SKA’s deep radio continuum surveys will provide dust-independent SF rate estimates, offering a crucial independent constraint to break degeneracies in IR-based AGN/SF decomposition. The combination of JWST and SKA data will thus help mitigate selection biases and enhance our understanding of AGN and SF co-evolution at high redshifts. Furthermore, these observatories will better observe and more accurately probe the nature of AGN and SF across cosmic time and the feedback processes that influence their evolution.

Additionally, improvements in SED decomposition algorithms, such as refined AGN torus models and machine learning-based fitting techniques, will enhance the accuracy of AGN-SF separation. Expanding the wavelength coverage to include extended far-IR observations will further improve constraints on dust-obscured star formation. Future studies should also adopt robust statistical methods, such as Bayesian inference and Monte Carlo simulations, to better quantify uncertainties in LFs and mitigate biases introduced by sample selection and luminosity estimation. These developments will help refine our understanding of galaxy evolution and the co-evolution of AGN and star formation across cosmic time. Follow-up studies utilising far-IR wavelength bands will be able to probe the obscured population of dusty active galaxies and AGN more accurately. This could be achieved with the JWST.

Envisioning the next generation of deep-field surveys and SED models also offers a path toward overcoming the key limitations identified in this thesis. Future deep-field surveys should combine ultra-deep, multi-wavelength coverage from X-ray to radio with wide-area mapping to simultaneously probe faint, obscured populations and build statistically robust samples across a range of environments and cosmic epochs. These surveys should aim to integrate high-resolution spectroscopy (e.g., with JWST NIRSpec) to provide precise redshifts and emission-line diagnostics critical for disentangling AGN and SF activity. Additionally, new SED models should incorporate more physically motivated and flexible AGN torus geometries, variable dust attenuation laws, and radiative transfer effects, particularly for composite systems. The development of data-driven SED fitting tools, such as those incorporating machine learning or neural networks trained on cosmological simulations, may offer breakthroughs in resolving highly degenerate solutions. Together, these advances would enable a more complete census of AGN and star-forming activity across cosmic time, refining our understanding of galaxy growth, black hole fueling, and the physical processes driving co-evolution.

\section*{Data Availability}
Python notebooks and scripts that analysed the data are available on GitHub at \url{https://github.com/daniel-lyon/MPhil-Code}

\newpage
\section*{Appendix A} \label{Sec: Appendix}

\begin{figure}[h!]
    \includegraphics[width=0.95\textwidth]{Figures/LF_Appendix.png}
    \caption*{The total IR $(8-1000\mu m)$ LF of ZFOURGE (Blue), CIGALE SF (Green), CIGALE AGN (Red), CIGALE Total (Orange).}
    \label{Fig: Appendix LF}
\end{figure}