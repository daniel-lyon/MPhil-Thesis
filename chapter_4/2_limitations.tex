\section{Limitations of the Thesis} \label{Sec: Limitations}

Despite the significant insights this thesis provided into the \gls{ir} \gls{lf} of \gls{sf} galaxies \gls{agn}, several limitations must be acknowledged.

\subsection{Sample Selection Bias and Incompleteness}
One primary concern is the potential incompleteness of the sample due to the selection criteria employed. As noted in paper reviewer comments, the NIR selection of the \gls{zfourge} survey predominantly samples the optical rest-frame at high redshift, favouring the detection of dust-free objects. This contradicts the goal of studying dusty \gls{ir} sources, particularly at high redshifts where the fraction of heavily obscured galaxies is expected to be higher \citep{symeonidis_agn_2021}. The exclusion of such sources may introduce an inherent bias in the derived LFs, potentially underestimating the number density of dusty, IR-luminous AGN and SF galaxies at high redshifts \citep{gruppioni_herschel_2013}.

\subsection{AGN and SF Decomposition Assumptions}
Removing AGN-dominated sources from the ZFOURGE total and CIGALE SF samples to prevent AGN contamination might inadvertently exclude sources where SF and AGN activity coexist \citep{hatziminaoglou_hermes_2010}. This could lead to an underestimation of the proper SF and AGN contributions. However, this underestimation is minimal and does not significantly influence the shape of the LF.

Furthermore, the AGN fraction cutoff for the decomposed AGN LF could still miss low-luminosity AGN, particularly those with significant star formation. \cite{hopkins_cosmological_2008} suggested that AGN activity can coexist with intense SF in specific environments. A more inclusive approach to AGN selection might provide a more comprehensive picture of their contribution to the LF. However, detecting an \gls{agn} fraction less than 10\% of the total luminosity is difficult. This is due to the limitations of \texttt{CIGALE} and the accuracy of decomposition in general \citep{hayward_should_2015}.

\subsection{Lack of Long-Wavelength Constraints}
The limited availability of far-IR photometry, particularly Herschel/SPIRE data at 250–500 $\mu$m, poses another constraint on the accuracy of the \gls{sed} fits. The reliance on Spitzer 24 $\mu$m and PACS 160 $\mu$m photometry may introduce systematic uncertainties in the estimated IR luminosities. As noted in the paper reviewer comments, at high redshifts, methods such as \cite{wuyts_fireworks_2008} tend to overestimate total IR luminosities due to limited sampling of the IR peak. The absence of direct far-IR constraints can affect the reliability of SF luminosity estimates and, consequently, the derived SF luminosity functions.

\subsection{Uncertainties in Luminosity Function Fitting}
The choice of different functional forms for fitting the various LFs (Schechter for SF, Saunders for AGN) may complicate direct comparisons between the SF and AGN LF. While the literature supports the use of the Schechter function for SF-dominated LFs \citep{wu_mid-infrared_2011} and the Saunders function for AGN \citep{symeonidis_agn_2021}, applying a consistent functional form across all components might facilitate more straightforward comparisons and minimize potential inconsistencies in parameter interpretations.

\subsection{Conclusion}
While this study provides a valuable analysis of decomposed IR LFs for SF galaxies and AGN, several limitations must be considered. These include biases in sample selection and a lack of far-IR photometry. Addressing these limitations in future work will be crucial for refining our understanding of the role of AGN and SF in galaxy evolution.