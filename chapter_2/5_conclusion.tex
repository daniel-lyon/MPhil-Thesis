\section{Conclusion}
Our tests in this chapter show that the methodology for creating \gls{ir} \gls{lf} is sound. The evolution of the different magnitude and spectral luminosity functions can be visualised across space and time. From fitting the Schechter function, it is clear that the evolution depends on both luminosity and redshift \citep{wu_mid-infrared_2011}. We do not analyse in depth the evolution of the two types of \gls{lf} introduced in this chapter, instead \Cref{Sec: Paper Chapter} delves into the significance of the total IR \gls{lf} and analyses the parameter evolution (\cref{Sec: Parameter Evolution}), luminosity density \cref{Sec: IR Density}, and class evolution (\cref{Sec: Class Density}).

Furthermore, \texttt{CIGALE} will be introduced and the methodology used to decompose the \gls{zfourge} survey \citep{straatman_fourstar_2016} into \gls{sf} and \gls{agn} components. A key element is the investigation of the simultaneous \gls{sf} evolution and \gls{agn} co-evolution with the host galaxy. This will allow us to place essential constraints on galaxy evolution across most of the universe's history.