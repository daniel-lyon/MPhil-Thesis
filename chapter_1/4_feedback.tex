\subsection{Gas Flows \& Feedback Mechanisms} \label{Sec: Feedback}
The formation of stars, galaxies, and \gls{agn} necessitates substantial material to support their development. Consequently, the evolution of AGN is intrinsically linked to \gls{sf} as the same interstellar material fuels the activity of both \citep{huang_local_2007, hopkins_cosmological_2008, biviano_spitzer_2011, katsianis_evolution_2017, symeonidis_agn_2021}. This relationship has sparked an ongoing debate: Does AGN activity enhance or inhibit \gls{sf}? \citep{fiore_agn_2017}. While there is a considerable body of research on this subject \citep{merloni_synthesis_2008, hopkins_how_2010, schawinski_observational_2007, cicone_massive_2014, reines_relations_2015, kauffmann_host_2003, silk_unleashing_2013, cowley_zfourge_2016}, the precise impact of AGN activity on galaxy evolution remains poorly understood \citep{fiore_agn_2017}. 

\subsubsection{AGN Feedback}
As noted by \cite{hopkins_cosmological_2008}, the most luminous AGN are frequently observed in conjunction with a significant, short-lived starburst phase lasting approximately 100 Myr. This could arise by a large influx of gas towards the galaxy's core \citep{hopkins_how_2010}, often referred to as the quasar or radiative mode, or it may result from AGN jets produced by a highly magnetised, rotating accreting BH, known as jet mode \citep{blandford_relativistic_2019}. Both processes require large quantities of gas and can influence the host galaxy through feedback mechanisms. The feedback of an \gls{agn} manifests as outflows that may positively or negatively affect \gls{sf}. 

In negative feedback mode, powerful outflows can potentially expel the interstellar medium from the galaxy if the feedback is strong enough \citep{fiore_agn_2017, schawinski_observational_2007, cicone_massive_2014}. This can result in the cessation of both \gls{sf} and BH growth, ultimately resulting in a quiescent galaxy \citep{hopkins_how_2010}. This phenomenon is referred to as the ``blowout" phase. It occurs when an AGN completely expels its accreting material due to overwhelming outward pressure, typically associated with the most luminous \gls{agn} \citep{hickox_obscured_2018}. Furthermore, \cite{mcnamara_heating_2007} notes that X-rays emitted from AGN contribute to maintaining high temperatures in the interstellar medium, thereby suppressing \gls{sf}. Additionally, the jets can cause shock heating in the circum-galactic medium which prevents material from cooling and moving in towards the ISM to form new stars \citep{heckman_nature_1990}. 

In positive feedback mode, as reported by \cite{silk_unleashing_2013}, less intense AGN-driven winds can also stimulate new \gls{sf}. This may be attributed to the consolidation of gas and dust. In some systems, a self-regulating feedback cycle is established, where AGN feedback is intermittently activated and deactivated. When active, radiative winds propel the gas away from the AGN, leading to its deactivation. Once the gas cools and migrates inward, the cycle repeats \citep{heckman_coevolution_2014}.

The dust-obscured AGN population is essential in AGN feedback models \citep{gruppioni_modelling_2011, hickox_obscured_2018}. Increased gas and dust correlate with a more powerful AGN, and feedback processes are more pronounced \citep{hickox_obscured_2018}. Conversely, unobscured AGNs are in the transition phase to becoming quenched galaxies \citep{hopkins_cosmological_2008}. However, \cite{frias_castillo_at_2024} found some of these galaxies still contain significant quantities of cool gas, indicating an elevated level of \gls{sf}. These observations imply that \gls{agn} activity may cease before \gls{sf} halts, although this isn't always the case \citep{morganti_many_2017, cielo_agn_2018}. The detailed interplay of feedback mechanisms associated with \gls{agn} presents a highly complex landscape that influences cosmic structures in both positive and negative ways. Despite substantial research efforts to explain the nuances of \gls{agn} feedback, the specific mechanisms underlying these processes remain poorly understood \citep{fiore_agn_2017}.

\subsubsection{Supernovae Feedback}
Feedback from AGN is not the sole mechanism for regulating \gls{sf} in galaxies. Supernova feedback is particularly significant, especially at higher redshifts when the universe was more compact \citep{heckman_nature_1990, katsianis_evolution_2017}. This observation is consistent with reviews on supernovae and feedback processes in the early universe conducted by \cite{volonteri_assembly_2003, cayrel_first_2004, sokasian_cosmic_2004, klessen_first_2023}. Although it is still under debate, it is generally believed that feedback from the first population III stars that experienced core-collapse supernovae created cosmic winds in the condensed universe which drove surrounding gas outward. \cite{croton_many_2006} demonstrates that these initial massive stars in galaxies will become the first supernovae, reheating the cold gas reservoir and suppressing localised \gls{sf}. Similar to AGN, research by \cite{sokasian_cosmic_2004} (and references within) indicates that the pressure waves generated by supernovae can compress gas towards the galactic halo, thereby aiding the formation of new stars.

\subsubsection{Mergers}
Mergers of smaller galaxies are thought to significantly contribute to the rapid growth of black holes on short timescales \citep{kormendy_coevolution_2013}—these mergers channel material toward the galaxy's core, activating the AGN. Analysis by \cite{hopkins_cosmological_2008} indicates most mergers occur in over-dense clustering regions. This follows intuitively from prominent \gls{lss} evolution theory; few mergers are expected to appear in cosmic voids \citep{coil_large-scale_2013}. It is through the merging of smaller dwarf galaxies in the earlier universe that initially seeded the accretion of matter onto SMBHs we observe today \citep{forbes_keckkcwi_2024, ziparo_primordial_2024} and their gas-rich content, which activates the AGN \citep{hopkins_how_2010, hartzenberg_evolved_2023}. Notably, significant mergers of larger galaxies have also been shown to enhance \gls{sf} (\citealp{volonteri_assembly_2003, hopkins_how_2010} and references within). The feedback dynamics and the various triggering mechanisms, such as mergers and gas inflows, are complex and can vary significantly between galaxies. Even within different types of elliptical galaxies, other forms of feedback are observed \citep{kormendy_coevolution_2013}, suggesting that a galaxy's local environment can profoundly influence the feedback mode. 

\subsubsection{IR SF-AGN Connection}
IR light is the ideal regime for studying dust obscured objects because these sources absorb most of the outgoing X-ray, Optical, and UV wavelengths and re-emit their radiation in the IR domain \citep{wu_mid-infrared_2011, toba_9_2013, oconnor_luminosity_2016}. As found by \cite{valiante_backward_2009} in the local universe, galaxies with higher \gls{sf} tend to have greater \gls{ir} luminosities. However, it has become apparent that as IR luminosity increases, \gls{agn} activity also increases, establishing IR as a strong indicator of \gls{agn} activity \citep{huang_local_2007, biviano_spitzer_2011, katsianis_evolution_2017, symeonidis_agn_2021}. \gls{ir} emission probes both \gls{sf} and \gls{agn} activity \citep{fu_decomposing_2010}. \cite{symeonidis_agn_2021} argue that the positive correlation between IR luminosity and \gls{sf} must dissolve at higher IR luminosity where \gls{agn} activity begins increasing with IR luminosity. This raises the possibility that many infrared galaxies currently identified as extreme starbursts may be influenced by contaminating \gls{agn} light \citep{symeonidis_agn_2021}. \gls{agn} are \textit{necessary} to observe the most luminous galaxies ($L_{IR} > 10^{13} \ L_\odot$) in our universe because such galaxies are \gls{agn} dominated. Without \gls{agn}, such bright galaxies would be extremely difficult to detect. While an increase in IR luminosity generally correlates with heightened \gls{agn} activity, high \gls{agn} activity has also been observed to follow equally intense \gls{sf} \citep{hopkins_cosmological_2008}, potentially due to feedback effects.

An essential factor influencing the relationship between \gls{sf} and \gls{agn} activity is redshift. In the local universe, the findings from various authors are conflicting. For instance, \cite{katsianis_evolution_2017} reports that \gls{agn} feedback is more prominent at lower redshifts, which subsequently lowers \gls{sf}. In contrast, \cite{fu_decomposing_2010} argues that \gls{lirg} ($L_{IR} > 10^{11} \ L_\odot$) host the majority of active \gls{sf} at higher redshift regimes. However, as discussed, there is a positive correlation between IR luminosity and \gls{agn} contamination: findings by \cite{wu_mid-infrared_2011} show that \gls{sf} decreases with increasing redshift. This finding contradicts the conclusions of \cite{fu_decomposing_2010}; however, this discrepancy may stem from differing interpretations of what constitutes ``high-redshift" among the authors. \cite{wu_mid-infrared_2011} investigates the redshift range $z<0.3$ while \cite{fu_decomposing_2010} from $z\approx0.7$. It is increasingly understood today that the \gls{agn} fraction is positively correlated with \gls{ir} luminosity and redshift \citep{symeonidis_agn_2021}. 

The findings suggest that \gls{agn} activity was higher in the early universe but has decreased in the local universe. The decline in \gls{agn} activity is thought to be correlated with a decrease in \gls{sf} \citep{fanidakis_evolution_2012}. As demonstrated by \cite{madau_cosmic_2014}, the \gls{sfr} peaks at a redshift of $z \approx 2$ before declining to present-day levels. \gls{agn} may exhibit the same decline as seen in \citep{aird_evolution_2015, symeonidis_agn_2021}. In the local universe, feedback effects from AGN may become more pronounced as galaxies become increasingly quenched \citep{frias_castillo_at_2024}. Additionally, AGN feedback likely becomes more significant below $z \approx 1$ \citep{katsianis_evolution_2017}.

\subsubsection{Conclusions}
The results from the literature paint a compelling story: \gls{agn} were very active in the early universe, and so too was \gls{sf}. However, that no longer appears to be the case in the local universe. Instead, \gls{agn} activity may inhibit \gls{sf}. One potential solution is that the early universe had abundant gas to feed both processes. However, as the gas supply became increasingly depleted, \gls{agn} and \gls{sf} needed to compete for an increasingly exhausted fuel supply. \Cref{Sec: Spectral Energy Distributions} discusses how to disentangle \gls{sf} from \gls{agn}. This is compounded by the expansion of the universe and the fusion of lighter elements (such as hydrogen and helium) into heavier elements (such as beryllium and lithium) \citep{bromm_formation_2002}. This theory would explain why \cite{katsianis_evolution_2017} finds high \gls{agn} feedback in the local universe and why \gls{sf} and \gls{agn} appear active in the early universe \citep{kormendy_coevolution_2013, heckman_coevolution_2014, king_powerful_2015, blandford_relativistic_2019}.

Overall, the discussion surrounding the interpretation of \gls{ir} observations related to \gls{sf} and \gls{agn} activity is complex, mainly due to ongoing debates over confusing and conflicting results \citep{grazian_galaxy_2015}. The relationship between \gls{agn} activity, feedback, and \gls{sf} requires further investigation. Exactly how AGN activity and feedback processes like supernovae impact galaxy evolution and the complex interplay between the processes regulating \gls{sf} is poorly understood \citep{grazian_galaxy_2015}. However, our understanding is improving. Future observations using high-resolution telescopes, such as the upgrades to ALMA \citep{carpenter_alma_2023}, the opening of SKA \citep{dewdney_square_2009}, and the ongoing operations of \gls{jwst} \citep{gardner_james_2006}, will be able to more precisely probe the underlying conditions and nature of \gls{agn} feedback.