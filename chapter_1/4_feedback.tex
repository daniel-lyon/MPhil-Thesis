\subsection{Gas Flows \& Feedback Mechanisms} \label{Sec: Feedback}
The formation of stars, galaxies, and \gls{agn} necessitates substantial material to support their development. Consequently, the evolution of AGN is connected to \gls{sf} as the same interstellar material fuels the activity of both \citep{huang_local_2007, hopkins_cosmological_2008, biviano_spitzer_2011, katsianis_evolution_2017, symeonidis_agn_2021}. This relationship has sparked an ongoing debate: Does AGN activity enhance or inhibit \gls{sf}? \citep{fiore_agn_2017}. There exists a wealth of research on this topic \citep{merloni_synthesis_2008, hopkins_how_2010, schawinski_observational_2007, cicone_massive_2014, reines_relations_2015, kauffmann_host_2003, silk_unleashing_2013, cowley_zfourge_2016}, but unfortunately the precise impact of AGN activity on galaxy evolution remains poorly understood \citep{fiore_agn_2017}. Yet, there is evidence to suggest a dependency on AGN activity. As noted by \cite{hopkins_cosmological_2008}, the most luminous AGNs are associated with significant \gls{sf} that can trigger intense AGN activity. This activity manifests as AGN-driven outflow winds that may expel the interstellar medium (ISM) from the galaxy \citep{fiore_agn_2017, schawinski_observational_2007, cicone_massive_2014}, potentially resulting in the starvation of the galaxy and the cessation of both \gls{sf} and BH growth \citep{hopkins_how_2010}. Conversely, as reported by \cite{silk_unleashing_2013}, AGN-driven winds can also stimulate new \gls{sf}. This may be attributed to the consolidation of gas and dust. 

The dust-obscured AGN population plays a crucial role in models of AGN feedback. Gas inflow, caused by friction within the accretion disk, facilitates AGN growth, leading to the formation of powerful jets emitted from the poles \citep{hickox_obscured_2018}. In contrast, unobscured AGNs, such as the quasars observed in the local universe ($z<1$) discussed by \cite{frias_castillo_at_2024}, exhibit very short gas depletion times, which indicate a high level of star formation activity. As discussed by \cite{mcnamara_heating_2007}, X-rays emitted from AGN jets contribute to maintaining hot temperatures in the interstellar medium, which suppresses star formation. In some systems, a self-regulating feedback cycle is established, where AGN feedback repeatedly activates and deactivates. While active, radiative winds propel the gas away from the AGN, deactivating it. After the gas cools and falls back inward, the cycle repeats \citep{heckman_coevolution_2014}. Additionally, a similar phenomenon, referred to as the ``blowout" phase, occurs when an AGN completely discards its accreting material supply due to overwhelming outward pressure \citep{hickox_obscured_2018}.

Mergers of smaller galaxies are thought to significantly contribute to the rapid growth of black holes on short timescales \citep{kormendy_coevolution_2013}, channelling material toward the galaxy's core and subsequently activating the AGN. Work by \cite{hopkins_cosmological_2008} reveals mergers occur in over-dense clustering regions. This follows intuitively from prominent \gls{lss} evolution theory; we do not expect many mergers to occur in cosmic voids \citep{coil_large-scale_2013}. It is through the mergers of smaller dwarf galaxies that seed the growth of SMBHs we observe today \citep{forbes_keckkcwi_2024, ziparo_primordial_2024} and their gas-rich content which activates the AGN \citep{hopkins_how_2010}. However, it has been shown that major mergers of larger galaxies contribute to enhancing \gls{sf} as well (\citealp{volonteri_assembly_2003, hopkins_how_2010} and references within). The role of feedback and the various triggering mechanisms, such as mergers and gas inflows, is complex and can differ significantly from one galaxy to another. Even within different types of elliptical galaxies, separate forms of feedback are observed \citep{kormendy_coevolution_2013}, suggesting that a galaxy's local environment can influence the mode of feedback. 

Feedback from AGN is not the only mechanism for regulating \gls{sf} in galaxies. Supernova feedback is particularly significant, especially at higher redshifts when the universe was more compact \citep{heckman_nature_1990, katsianis_evolution_2017}. This aligns with reviews on supernovae and feedback processes in the early universe conducted by \cite{volonteri_assembly_2003, cayrel_first_2004, sokasian_cosmic_2004, klessen_first_2023}. The feedback from the first population III stars that underwent core-collapse supernovae and their subsequent winds in the more condensed universe drives the surrounding gas outward. \cite{croton_many_2006} illustrates that initial massive stars in galaxies will become the first supernovae, reheating the cold gas reservoir and suppressing localised \gls{sf}. Similar to AGN, research by \cite{sokasian_cosmic_2004} (and references within) indicates that the pressure waves generated by supernovae can compress gas towards the galactic halo, thereby aiding the formation of new stars.

\subsubsection{Infrared}

As found by \cite{valiante_backward_2009} in the local universe, galaxies with higher \gls{sf} tend to have greater IR luminosities. Authors \cite{symeonidis_agn_2021} find the positive correlation between IR luminosity and \gls{sf} must dissolve at higher IR luminosity where \gls{agn} activity begins increasing with IR luminosity. Indeed, \gls{ir} emission probes both \gls{sf} and \gls{agn} activity \citep{fu_decomposing_2010}. However, as IR luminosity increases, \gls{agn} activity also increases, making IR a strong indicator of \gls{agn} activity \citep{symeonidis_agn_2021, biviano_spitzer_2011, huang_local_2007, katsianis_evolution_2017}. It is possible that many IR galaxies currently considered to be extreme star-burst could be inflated by contaminating \gls{agn} light \citep{symeonidis_agn_2021}. \gls{agn} are \textit{necessary} to observe the most luminous galaxies ($L_{IR} > 10^{13} \ L_\odot$) in our universe because such galaxies are \gls{agn} dominated \citep{symeonidis_agn_2021}. Without \gls{agn}, such bright galaxies would be extremely difficult to detect. Even though increasing IR luminosity positively correlates with AGN activity, high \gls{agn} activity has also been observed to follow equally intense \gls{sf} \citep{hopkins_cosmological_2008, katsianis_evolution_2017}. Overall, the discussion surrounding the interpretation of \gls{ir} observations related to \gls{sf} and \gls{agn} activity is complex, largely due to ongoing debates over conflicting results \citep{grazian_galaxy_2015}. 

An important factor influencing the relationship of \gls{agn} activity is redshift. According to \cite{fu_decomposing_2010}, \gls{lirg} host the majority of active \gls{sf} at higher redshift regimes. However, as discussed, there is a positive correlation between IR luminosity and \gls{agn} contamination. \cite{wu_mid-infrared_2011} found that \gls{sf} decreases with increasing redshift. This finding appears to contradict the conclusions of \cite{valiante_backward_2009} and \cite{fu_decomposing_2010}; however, this discrepancy may stell from differing interpretations of what constitutes ``high-redshift" among the authors. On the other hand, \cite{katsianis_evolution_2017} reports that \gls{agn} feedback is more prominent at lower redshifts, which subsequently lowers \gls{sfr}s. From these findings, it can be inferred that \gls{agn} activity was more significant in the early universe but has diminished into the local universe. However, the decline in \gls{agn} activity does not necessarily correlate with a consistent increase in \gls{sf} \citep{fanidakis_evolution_2012}. As \cite{madau_cosmic_2014} shows, \gls{sfr} peaks at $z=2$ before declining to present levels. Nevertheless, \gls{agn} feedback --- distinct from \gls{agn} activity --- may potentially increase in the local universe, although the timescales for such feedback are not fully understood. \gls{agn} feedback likely increases from at least $z\approx0.8$ \citep{katsianis_evolution_2017}. 

The results from the literature paint a compelling story: \gls{agn} were super active in the early universe, and so too was \gls{sf}. However, that no longer appears to be the case in the local universe. Instead, \gls{agn} activity may now inhibit \gls{sf}. One potential solution is that the early universe had an abundance of gas that could feed both processes. However, as the gas supply became increasingly depleted, \gls{agn} and \gls{sf} needed to fight for an increasingly exhausted fuel supply. This is compounded by the expansion of the universe and the fusion of lighter elements favouring SF into heavier elements which cannot sustain SF. This theory would explain why \cite{katsianis_evolution_2017} finds high \gls{agn} feedback in the local universe and why \gls{sf} and \gls{agn} were so active in the early universe. 

The relationship between \gls{agn} activity, feedback, and \gls{sf} requires further investigation, as the literature is often confusing and conflicting. Exactly how AGN activity and feedback processes like supernovae impact galaxy evolution and the complex interplay between the processes regulating \gls{sf} is poorly understood \citep{grazian_galaxy_2015}. However, our understanding is improving. Future observations using high-resolution telescopes, such as the upgrades to ALMA \citep{carpenter_alma_2023}, the opening of SKA \citep{dewdney_square_2009}, and the ongoing operations of \gls{jwst} \citep{gardner_james_2006}, will be able to more precisely probe the underlying conditions and nature of feedback.