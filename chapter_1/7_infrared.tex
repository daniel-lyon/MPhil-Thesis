\subsection{Infrared}

\begin{itemize}
    \item \textcolor{red}{introduction to and history of luminosity functions}
    \item \textcolor{red}{review luminosity functions across multiple wavelengths and tie into how infrared is the best}
\end{itemize}

When analysing the \gls{lf} of IR galaxies, it is paramount that both the SF and AGN LF are analysed in parallel. Otherwise, crucial information regarding the underlying physical co-evolution between the processes will be omitted, and the wrong individual conclusions will be drawn. As was reviewed in \cref{Sec: Feedback}, \gls{ir} emission probes both \gls{sf} and \gls{agn} activity \citep{fu_decomposing_2010}. The literature contains limited research on \gls{agn} \gls{lf}s and even less on the IR regime. Consequently, the \gls{ir} \gls{agn} \gls{lf} represents a relatively unexplored approach to quantifying the co-evolution of galaxies and BHs. Most authors have concentrated on the galaxy or \gls{sf} \gls{lf}, which traces the evolution of galaxies \citep{cool_galaxy_2012, tempel_tracing_2011}. However, these studies often overlook the co-evolution with BHs \citep{fotopoulou_5-10_2016, symeonidis_agn_2021, finkelstein_coevolution_2022} that galaxies have been shown to depend on \citep{hopkins_cosmological_2008, fiore_agn_2017}.

One of the most commonly used methods for calculating \gls{lf} is with the Schechter function \citep{schechter_analytic_1976}. However, \cite{wu_mid-infrared_2011} reports that the mid-\gls{ir} wavelengths exhibit a shallower exponential profile inconsistent with a Schechter function. According to \cite{fu_decomposing_2010}, this discrepancy may be attributed to \gls{agn} contamination of the \gls{lf} and, when removed, can accurately describe a Schechter function. Indeed, this is the case for IR \gls{agn} \gls{lf}s, which are inconsistent with a Schechter function \citep{symeonidis_what_2019}. The Saunders function \citep{saunders_60-mum_1990} is typically used to model \gls{agn} \gls{lf}s. The properties of IR light make it the ideal regime for studying \gls{agn}. For example, IR light can detect unobscured (Type 1) and heavily obscured (Type 2) \gls{agn}. Type 2 sources absorb most of the outgoing X-ray, Optical, and UV wavelengths and re-emit their radiation in the IR domain \citep{fu_decomposing_2010, wu_mid-infrared_2011, gruppioni_modelling_2011, assef_mid-ir-_2011, toba_9_2013, oconnor_luminosity_2016, brown_infrared_2019, symeonidis_agn_2021}. The ability to detect both types of \gls{agn} is a significant factor influencing the decision to study IR \gls{lf}s, as other wavelengths struggle to identify heavily obscured sources. However, as noted by \cite{katsianis_evolution_2017}, infrared observations are primarily effective for dusty, massive galaxies and become limited at higher redshifts. Additionally, less obscured (Type 1) \gls{agn} may appear dimmer in the IR because they emit less light from reprocessing other wavelengths around the dusty torus. 

Results by \cite{han_evolution_2012} show the faint end slope of the \gls{lf} flattens with increasing redshift. However, this may be due to poor sampling of the faint end, which could be a binning effect or Malmquist bias, as discussed previously \citep{madau_cosmic_2014}. Splitting the \gls{lf} into redshift and luminosity bins has the unfortunate side effect of reducing sample sizes, leading to an increase in errors and error propagation \citep{dai_mid-infrared_2009}. However, it is also necessary because it allows for the evolution to be studied across cosmic time \citep{wu_mid-infrared_2011, wylezalek_galaxy_2014}. It remains possible that surveys designed to probe fainter luminosities at greater redshift will uncover many smaller ``building block-style" dwarf galaxies as current leading theories of cosmology and \gls{lss} predict \citep{magorrian_demography_1998, ziparo_primordial_2024}. Although, as was reviewed in \cref{Sec: First Stars}, results by \cite{labbe_population_2023} have cast doubt on these theories.

The evolution of different luminosity classes with redshift can be directly derived from the changes in the \gls{lf}. \cite{han_evolution_2012} observed a \textit{downsizing} effect in their study of \gls{agn}, indicating that brighter galaxies begin declining in number density at higher redshifts compared to than fainter \gls{agn} counterparts. Moreover, \cite{merloni_synthesis_2008} found a reversal in the downsizing effect above $z=2$, coinciding with the peak of cosmic \gls{sf} rate density. This is likely because the authors focus on the kinetic \gls{lf} of \gls{agn} jets and not \gls{agn} as a whole. Although \cite{fanidakis_evolution_2012} demonstrated clear downsizing of the entire \gls{agn}, which is irreconcilable with \cite{merloni_synthesis_2008}: how can there be more \gls{agn} jets than \gls{agn}? \cite{fiore_agn_2017} speculates that the downsizing effect may originate from feedback mechanisms. \cite{wylezalek_galaxy_2014} proposes that the oldest stellar populations reside in massive clusters, the likes of which are not seen in the local universe \citep{croton_many_2006}. Intuitively, galaxy clusters likely support stronger \gls{sf}, but \cite{wang_imperial_2010} shows clusters are positively correlated with \gls{ir} luminosity, which \cite{symeonidis_agn_2021} indicated could be increasingly contaminated by \gls{agn}. Similarly, for \gls{sfg}, results from \cite{gruppioni_herschel_2013} also revealed a \textit{downsizing} effect. It remains unclear whether the downsizing effects in \gls{sfg} and \gls{agn} are fundamentally different.

% The unified \gls{agn} model states there are two types of \gls{agn} \cite{toba_9_2013}. Type 1 \gls{agn}s are characterized by the presence of broad emission lines in their spectra and show an unobstructed view along a direct line of sight to the centre. Type 2 \gls{agn}s, on the other hand, exhibit only narrow emission lines in their spectra and are associated with an obscured line of sight to the centre. A direct view of the broad-line region is blocked or heavily absorbed by intervening material, such as the dusty torus. As a result, X-ray and UV wavelengths are utilised (cite) but omit the vast majority of Type 2 \gls{agn} galaxies from their analysis. This is a major factor influencing the decision to study IR \gls{lf}s as IR can detect both Types of \gls{agn} whereas other wavelengths cannot, or at best, struggle greatly. Furthermore, the dusty torus surrounding the black hole absorbs most of the outgoing optical, UV, and X-ray radiation that is subsequently re-emitted in the IR regime \citep{fu_decomposing_2010, assef_mid-ir-_2011, wu_mid-infrared_2011, han_evolution_2012, toba_9_2013, brown_infrared_2019, symeonidis_agn_2021}, again highlighting the usefulness of IR light over other wavelengths. Some research exists that challenges the unified \gls{agn} model and suggests the reality of \gls{agn} obscuration is far more complex than simple viewing angles might have us believe. \cite{han_evolution_2012} concluded that the obscuration resulting from the dusty torus must evolve over time. Indeed, \cite{brown_infrared_2019} agrees, finding that the line of sight towards \gls{agn}s is not the deciding factor on the resulting \gls{agn} type. The consequence of this ambiguity between whether different \gls{agn} types even exist means it is important to keep one consolidated \gls{agn} IR \gls{lf} instead of separating into two isolated \gls{lf}s, one for each type respectively.

% The Schechter function is particularly useful for describing the luminosity function of galaxies because it captures the observed features, such as a power-law decline at the faint end and an exponential cutoff at the bright end. Yet, there is a dilemma: \cite{wu_mid-infrared_2011} reports that the UV and optical wavelengths follow a Schechter function, but the IR wavelengths have a shallower exponential inconsistent with a Schechter function. \cite{fu_decomposing_2010} proposes that this is due to \gls{agn} contribution to the \gls{lf}, and when removed, can fit a Schechter function normally. Indeed, there is much agreement in the literature that the observed IR light contains both SF and \gls{agn} components (cite). Interestingly, though, few sources separate the SF and \gls{agn} light (cite), mostly because, until recently, this was very difficult to accomplish accurately (cite).

% The ZFOURGE survey offers a unique advantage by investigating galaxies at higher redshifts and fainter luminosities, leading to a deeper understanding of galaxy-BH co-evolution through discoveries at lower luminosities \citep{straatman_fourstar_2016}. By combining ZFOURGE data with \gls{lf}s, more powerful constraints can be placed on galaxy-BH co-evolution.