\subsection{Multiwavelength Considerations}
\gls{ir} observations of \gls{sf} and \gls{agn} aren't the only way of calculating \gls{lf}. Often, a multi-wavelength approach is taken \citep{marshall_decomposing_2007}. While \gls{ir} observations are some of the most popular due to the ability to accurately measure \gls{sf}, other wavelengths are more suited to probing other physical properties of galaxies. For example, the radio wavelengths are well suited to detecting the enormous polar jets extending from \gls{agn} \citep{kaiser_luminosity_2007, mauch_radio_2007}. X-ray radiation may be associated with \gls{agn} feedback and activity \citep{la_franca_tools_2010}. UV is more sensitive to \gls{sf} than \gls{ir}, but dust obscures most UV observations \citep{malefahlo_deep_2022}. The complex interplay between the various physical processes in galaxies allows for a multi-wavelength approach to isolating specific galaxy populations. Such populations of objects can be unique to certain wavelength regimes or more easily detected in others. This leads to an inherent bias in selection effects, further explained in the following sections.

\subsubsection{Radio}
Radio wavelengths may also be excellent tracers of \gls{sf} due to negligible dust obscuration, but \gls{agn} can contaminate samples \citep{yuan_mixture_2017, malefahlo_deep_2022}. \cite{yuan_mixture_2017} (and references within) show results that radio-loud \gls{agn} undergo significant evolution out to high redshift, whereas lower-luminosity counterparts do not evolve nearly as quickly or long. This is representative of downsizing, as discussed in the \gls{ir} regime in \cref{Sec: Intro Downsizing} and has significant implications for the radio \gls{lf}. Discussion by \cite{rigby_luminosity-dependent_2011} points out that radio observations can detect the largest and oldest \gls{agn} in the universe. Bright radio galaxies are increasingly likely to contain higher levels of \gls{agn} activity \citep{sadler_radio_2002}. Such information about the radio regime and its \gls{lf} is critical to uncovering the evolution of the most distant objects in the universe.

\subsubsection{X-ray}
Several studies on the X-ray \gls{lf} of obscured and unobscured \gls{agn} \gls{lf} have been published. \cite{aird_evolution_2015} lists the positives and negatives of X-ray observations to study all types of \gls{agn}. Most importantly, they find soft X-ray emission will still be obscured by the same interstellar dust and gas that surround the brightest and most active systems. However, this has the advantage of splicing the sample into unobscured (soft) and obscured (hard) X-ray observations. \cite{ananna_bass_2022} notes that hard X-ray observations are more reliable at detecting \gls{agn} than in the \gls{ir} due to X-rays occurring less in stellar emission. In terms of the \gls{lf}, X-ray observations are more suited to studies focusing on the differences between obscured and unobscured \gls{agn}. These two types are likely at different evolutionary stages, with unobscured \gls{agn} closer to quiescence \citep{frias_castillo_at_2024}. 

\subsubsection{Optical/UV}
Discussion by \cite{finkelstein_coevolution_2022} points out that most galaxies selected for observation are done so based on their UV-optical emission. Although \cite{finkelstein_coevolution_2022} also argues that obscuration effects do not play an important role at higher redshift regimes. However, this is very controversial because clusters are some of the oldest stellar populations \citep{wylezalek_galaxy_2014} and play an important role in galaxy evolution \citep{croton_many_2006, wang_bright_2021}. Recently \cite{labbe_population_2023} has shown galaxies occurring in the distant early universe with extreme dust levels. The most appropriate use of \gls{uv} wavelengths is likely found in low-redshift studies, e.g. \cite{jurek_wigglez_2013} who study the \gls{uv} \gls{lf} of galaxies between $0.1<z<0.9$. Although, high-redshift \gls{uv} \gls{lf} studies have been published, such as \cite{kulkarni_evolution_2019, finkelstein_coevolution_2022} from $0<z<7.5$ and $3<z<9$ respectively. Results from these studies should be taken with precaution due to their inherent bias in excluding obscured sources, which comprise the bulk of \gls{sf} and \gls{agn} activity \citep{gruppioni_modelling_2011, toba_9_2013}. The \gls{uv} \gls{lf} of galaxies and \gls{agn} have led to significant evolutionary constraints. However, it is advised only to utilise \gls{uv} wavelengths to study unobscured late-type \gls{lf} because this regime is not suited for the heavy dust-obscured population of galaxies and \gls{agn}.  

\subsubsection{Conclusion}
As examined, multi-wavelength \gls{lf} have led to powerful constraints placed on galaxy and \gls{agn} evolution models and theories. The various wavelength regimes host differing use cases, which all have flaws, biases, selection effects, advantages, and conveniences unique to the problem at hand. Each provides a method of selecting different classes of objects and dividing samples intuitively into comprehensible, sometimes exclusive populations that no other wavelength can achieve. Each radio, optical, and X-ray \gls{lf} are specialised. It is essential to consider all wavelength regimes to reach a comprehensive consensus on galaxy evolution and \gls{agn} coevolution.