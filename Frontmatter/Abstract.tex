\chapter{Abstract}
\thispagestyle{empty}

This study presents a comprehensive analysis of the infrared (IR) luminosity functions (LF) of star-formation (SF) and active galactic nuclei (AGN) using data from the ZFOURGE survey. We employ CIGALE to decompose the spectral energy distribution (SED) of galaxies into SF and AGN components to investigate the co-evolution of these processes at higher redshifts and fainter luminosities. Our CIGALE-derived SF and AGN LFs are generally consistent with previous studies, with an enhancement at the faint end of the AGN LFs. We attribute this to CIGALE's capability to recover low-luminosity AGN more accurately, which may be underrepresented in other works. We find evidence for a significant evolutionary epoch for AGN activity below $z \approx 2$, comparable to the peak of cosmic star formation at $z \approx 2$, which we also recover well. Based on our results, the gas supply in the early universe favoured the formation of brighter star-forming galaxies from high-redshift until $z=2$, below which the gas for SF becomes increasingly exhausted. In contrast, AGN activity peaked earlier and declined more gradually, suggesting a possible feedback scenario in which AGN positively influence SF. 

\vspace{1cm}
\noindent
\textbf{Keywords:} galaxies: luminosity function, mass function; cosmology: observations; infrared: galaxies; galaxies: evolution; active galactic nuclei